\title{Sample Title}
\subtitle{Sample Subtitle}
\author{Author}

% switch between styles
% (or change colors in beamercolorthememinima.sty)
\usebluestyle
% \useredstyle
% \usegraystyle

\begin{document}

	% produce secondary title page
	\setsecondayrtitlepage
	\begin{frame}
		\titlepage
	\end{frame}

	% produce primary title page
	\setprimarytitlepage
	\begin{frame}
		\titlepage
	\end{frame}

	\begin{frame}
		\frametitle{Simple Content}
		Here we have an \texttt{itemize} environment.
		\begin{itemize}
			\item First.
			\item Second.
			\item Third.
		\end{itemize}
		\vspace{1em}

		This is a paragraph. Lorem \textcolor{red}{ipsum} dolor sit amet, consetetur sadipscing elitr, sed diam nonumy \textcolor{green}{eirmod} tempor invidunt ut labore et dolore \textcolor{blue}{magna} aliquyam erat, sed diam voluptua. At vero eos et accusam et justo duo dolores et ea rebum. Stet clita kasd gubergren.
		\vspace{1em}

		Here we have an \texttt{enumerate} environment.
		\begin{enumerate}
			\item First.
			\item Second.
			\item Third.
		\end{enumerate}
	\end{frame}

	\begin{framefont}{\small}
	\begin{frame}
		\frametitle{Simple Content}
		\framesubtitle{\texttt{small}}
		Here we have an \texttt{itemize} environment.
		\begin{itemize}
			\item First.
			\item Second.
			\item Third.
		\end{itemize}
		\vspace{1em}

		This is a paragraph. Lorem \textcolor{red}{ipsum} dolor sit amet, consetetur sadipscing elitr, sed diam nonumy \textcolor{green}{eirmod} tempor invidunt ut labore et dolore \textcolor{blue}{magna} aliquyam erat, sed diam voluptua. At vero eos et accusam et justo duo dolores et ea rebum. Stet clita kasd gubergren.
		\vspace{1em}

		Here we have an \texttt{enumerate} environment.
		\begin{enumerate}
			\item First.
			\item Second.
			\item Third.
		\end{enumerate}
	\end{frame}
	\end{framefont}

	\begin{frame}
		\frametitle{Mathematical Content}
		\framesubtitle{An Example of Mathematical Content}

		\small
		\begin{align*}
            p(C, \Pi, \Theta, W) &= p( \Pi \mid \alpha) p(C \mid \Pi) p(\Theta \mid \beta) p(W \mid C, \Theta)\\
            &=
            \left(
            \prod_d^D p(\pi_d \mid \alpha_d)
            \right)
            \left(
            \prod_d^D
            \prod_i^{I_d} p(c_{di} \mid \pi_d)
            \right)
            \left(
            \prod_d^D
            \prod_i^{I_d} p(w_{di} \mid c_{di},\Theta)
            \right)
            \left(
            \prod_k^K
            p(\theta_k \mid \beta_k)
            \right)\\
            &=
            \left(
            \prod_d^D
            \ca D(\pi_d; \alpha_d)
            \right)
            \left(
            \prod_d^D
            \prod_i^{I_d}
            \left(
            \prod_k^K \pi_{dk}^{c_{dik}}
            \right)
            \right)
            \left(
            \prod_d^D
            \prod_i^{I_d}
            \left(
            \prod_k^K \theta_{kW_{di}}^{c_{dik}}
            \right)
            \right)
            \left(
            \prod_k^K
            \ca D (\theta_k; \beta_k)
            \right)\\
            &=
            \left(
            \prod_d^D
            \frac{\Gamma (\sum_k \alpha_{dk})}{\prod_k \Gamma(\alpha_{dk})}
            \prod_k^K \pi_{dk}^{\alpha_{dk}-1+n_{dk\cdot}}
            \right)
            \left(
            \prod_d^D
            \frac{\Gamma (\sum_v \beta_{kv})}{\prod_v \Gamma(\beta_{kv}))}
            \prod_v^V \theta_{kv}^{\alpha_{kv}-1+n_{\cdot kv}}
            \right)
        \end{align*}

        This is the joint probability for the Latent Dirichlet Allocation. You can find more information on this topic \hyperlink{https://www.youtube.com/watch?v=o22cA1DhSMQ&list=PL05umP7R6ij1tHaOFY96m5uX3J21a6yNd&index=22&t=0s}{here} (where also this formula is taken from).
	\end{frame}
	\begin{framefont}{\small}
	\begin{frame}
	    \frametitle{Mathematical Content}
	    \framesubtitle{Theorems and Proofs}

	    \begin{definition}[odd integer]
	    An integer $z \in \mathbb{Z}$ is said to be odd if it is not divisible by two, i.e. there exist no $k \in \mathbb Z$ s.t. $z = 2k$.
	    \end{definition}

	    \begin{theorem}[Multiplication of Odd Integers Yields Even Integer]
	    Let $a, b \in \mathbb Z$ be two non-null odd integers. Then $a \cdot b$ is an even integer.
	    \end{theorem}

	    \begin{proof}
	    Let $k, l \in \mathbb{Z} \backslash \{0\}, a = 2k+1, b = 2l+1$. Then
	    $$
	    a \cdot b = (2k+1)(2l+1) = 4kl + 2k + 2l + 2 = 2(2kl + k + l + 1)
	    $$
	    which is even.
	    \end{proof}
	\end{frame}
	\end{framefont}

	\begin{frame}
	    \frametitle{Algorithmic Content}
	    \framesubtitle{Bubble Sort}
	    Bubble Sort is an algorithm to sort an array of real numbers.
	    \vspace{1em}

	    \begin{figure}
    	    \centering
    	    \begin{algorithm}[H]
    	        \caption{\texttt{BubbleSort($A$)}}
        	    $n \leftarrow A.\texttt{length}$\\
        	    \For{$i=1$ to $n$}
        	    {
        	    \For{$j=0$ to $n-i$}
        	    {
        	    \If{$A[j+1] < A[j]$}
        	    {
        	    exchange values at positions $j+1$ and $j$ in $A$
        	    }
        	    }
        	    }
        	    \Return $A$
    	    \end{algorithm}
	    \end{figure}
	\end{frame}

	\begin{frame}
	    \frametitle{Blocks and Stripes}

	    \begin{block}{Important information.}
	    Some content is just too important to leave it without highlighting on a slide
	    \end{block}

	    \begin{stripe}
	    Some content is just too important to leave it without highlighting on a slide. Like \textcolor{red}{this} needs to be additionally highlighted.
	    \end{stripe}
	 \end{frame}

	 \begin{frame}[highlight]
	    \frametitle{Highlight Title}
	    \framesubtitle{Highlight Subtitle}
	    \centering
	    Sometimes there is a need for a special highlighting page to separate different topics in the presentation.
 	 \end{frame}

      \begin{frame}[highlight]
        \centering
	    Sometimes there is a need for a special highlighting page to separate different topics in the presentation.
	    \vspace{1em}
	    Even without a title.
 	 \end{frame}

	 \begin{frame}
		 \frametitle{Simple Content}
		 \framesubtitle{... after a highlight-slide}
		 Here we have an \texttt{itemize} environment.
		 \begin{itemize}
			 \item First.
			 \item Second.
			 \item Third.
		 \end{itemize}
		 \vspace{1em}

		 This is a paragraph. Lorem \textcolor{red}{ipsum} dolor sit amet, consetetur sadipscing elitr, sed diam nonumy \textcolor{green}{eirmod} tempor invidunt ut labore et dolore \textcolor{blue}{magna} aliquyam erat, sed diam voluptua. At vero eos et accusam et justo duo dolores et ea rebum. Stet clita kasd gubergren.
		 \vspace{1em}

		 Here we have an \texttt{enumerate} environment.
		 \begin{enumerate}
			 \item First.
			 \item Second.
			 \item Third.
		 \end{enumerate}
	 \end{frame}

\end{document}
